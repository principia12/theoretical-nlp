\documentclass[twoside]{article}

\usepackage{lipsum} % Package to generate dummy text throughout this template

\usepackage[sc]{mathpazo} % Use the Palatino font
\usepackage[T1]{fontenc} % Use 8-bit encoding that has 256 glyphs
\linespread{1.05} % Line spacing - Palatino needs more space between lines
\usepackage{microtype} % Slightly tweak font spacing for aesthetics

\usepackage[hmarginratio=1:1,top=32mm,columnsep=20pt]{geometry} % Document margins
\usepackage{multicol} % Used for the two-column layout of the document
\usepackage[hang, small,labelfont=bf,up,textfont=it,up]{caption} % Custom captions under/above floats in tables or figures
\usepackage{booktabs} % Horizontal rules in tables
\usepackage{float} % Required for tables and figures in the multi-column environment - they need to be placed in specific locations with the [H] (e.g. \begin{table}[H])
\usepackage{hyperref} % For hyperlinks in the PDF

\usepackage{lettrine} % The lettrine is the first enlarged letter at the beginning of the text
\usepackage{paralist} % Used for the compactitem environment which makes bullet points with less space between them

\usepackage{braket}
\usepackage{array}
\usepackage{calc}
\usepackage{graphicx}
\usepackage{listings}
\usepackage{color}
\usepackage[table,xcdraw]{xcolor}
\usepackage{adjustbox}


\definecolor{dkgreen}{rgb}{0,0.6,0}
\definecolor{gray}{rgb}{0.5,0.5,0.5}
\definecolor{mauve}{rgb}{0.58,0,0.82}



\hypersetup{%
    pdfborder = {0 0 0}
}



\usepackage{abstract} % Allows abstract customization
\renewcommand{\abstractnamefont}{\normalfont\bfseries} % Set the "Abstract" text to bold
\renewcommand{\abstracttextfont}{\normalfont\small\itshape} % Set the abstract itself to small italic text

\usepackage{titlesec} % Allows customization of titles
%\renewcommand\thesection{\Roman{section}} % Roman numerals for the sections
\renewcommand\thesubsection{\Roman{subsection}} % Roman numerals for subsections
\titleformat{\section}[block]{\large\scshape\centering}{\thesection.}{1em}{} % Change the look of the section titles
\titleformat{\subsection}[block]{\large}{\thesubsection.}{1em}{} % Change the look of the section titles
% \setcounter{section}{1}

\usepackage{fancyhdr} % Headers and footers
\pagestyle{fancy} % All pages have headers and footers
\fancyhead{} % Blank out the default header
\fancyfoot{} % Blank out the default footer
\fancyhead[C]{ NLP101 Theoretical Natural Language Processing} % Custom header text
\fancyfoot[RO,LE]{\thepage} % Custom footer text

%----------------------------------------------------------------------------------------
%	TITLE SECTION
%----------------------------------------------------------------------------------------

\begin{document}
\begin{titlepage}

\newcommand{\HRule}{\rule{\linewidth}{0.5mm}} % Defines a new command for the horizontal lines, change thickness here

\center % Center everything on the page

%----------------------------------------------------------------------------------------
%	HEADING SECTIONS
%----------------------------------------------------------------------------------------

\vspace*{3cm}
\textsc{\Large NLP101}\\[0.5cm] % Major heading such as course name
\textsc{\large Theoretical Natural Language Processing}\\[0.5cm] % Minor heading such as course title

%----------------------------------------------------------------------------------------
%	TITLE SECTION
%----------------------------------------------------------------------------------------

\HRule \\[0.4cm]
{ \huge \bfseries Survey on Syntax-Semantic Interface: from Formal Languages to Natural Languages}\\[0.4cm] % Title of your document
\HRule \\[1.5cm]

%----------------------------------------------------------------------------------------
%	AUTHOR SECTION
%----------------------------------------------------------------------------------------

\begin{minipage}{0.4\textwidth}
\begin{flushleft} \large
\emph{Author:}\\
Seungwoo \textsc{Schin} % Your name
\end{flushleft}
\end{minipage}
~
\begin{minipage}{0.4\textwidth}
\begin{flushright} \large
\emph{Typeset by:} \\
Seungwoo \textsc{Schin} % Supervisor's Name
\end{flushright}
\end{minipage}\\[4cm]

% If you don't want a supervisor, uncomment the two lines below and remove the section above
%\Large \emph{Author:}\\
%John \textsc{Smith}\\[3cm] % Your name

\textsc{ NCSOFT, Natural Language Generation Team}\\[1.5cm] % Name of your university/college

%----------------------------------------------------------------------------------------
%	DATE SECTION
%----------------------------------------------------------------------------------------

{\large \today}\\[3cm] % Date, change the \today to a set date if you want to be precise
% 2014 Fall Semester

%----------------------------------------------------------------------------------------
%	LOGO SECTION
%----------------------------------------------------------------------------------------

%\includegraphics{Logo}\\[1cm] % Include a department/university logo - this will require the graphicx package

%----------------------------------------------------------------------------------------

%\vfill % Fill the rest of the page with whitespace

\end{titlepage}

% Table of contents

\tableofcontents
\newpage



\lstset{frame=tb,
  language=lisp,
  aboveskip=3mm,
  belowskip=3mm,
  showstringspaces=false,
  columns=flexible,
  basicstyle={\small\ttfamily},
  numbers=none,
  numberstyle=\tiny\color{gray},
  keywordstyle=\color{blue},
  commentstyle=\color{dkgreen},
  stringstyle=\color{mauve},
  breaklines=true,
  breakatwhitespace=true
  tabsize=3
}

\begin{abstract}

The purpose of this paper is to inspect various models for the semantics of natural language

In this paper, we attempt to re-organize existing theories of synatx-semantic interface starting from the corresponding counterparts in formal languages. We begin by reviewing the definition of the language and various classes of languages from the perspective of formal language theory. proposed in Chomsky Hierarchy\cite{Chomsky1956}. To be specific, regular, context-free, context-sensitive languages will be discussed in the perspective of syntactic model. Thereafter, examples of regular and context-free languages will be discussed.
\end{abstract}

\section{Introduction : What is Natural Language Processing?}

There has been much research regarding syntax, semantic, and their relationships in natural language. Paradoxically, this shows the extreme vagueness and shallowness of our understadning on semantic. Various models have been proposed to illustrate the semantics of natural language, with their own meta-language to express the semantics. To list few examples, logical forms\cite{sep-logicalform}, Montague Semantics\cite{sep-montague}, Davidsonian and Neo-Davidsonian Event Semantics, real-valued vector\cite{mikolov2013} could be some of the examples.

\subsection{What is Language?}

\subsection{What Language is 'Natural'?}

\subsection{Why is Natural Language so HARD?}

\section{Syntax of Languages}

\section{Semantic of Languages}

% \section{




% \includegraphics[width=10cm]{PLA}
\bibliographystyle{apalike}
\bibliography{reference}

\end{document}



























